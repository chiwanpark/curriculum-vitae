%! TeX program = lualatex
\documentclass[11pt,a4paper]{article}
\usepackage[dvipsnames]{xcolor}
\usepackage[margin=0.8in]{geometry}
\usepackage{kotex}
\usepackage{hyperref}
\usepackage{enumitem}
\usepackage{parskip}
\usepackage{titlesec}
\usepackage{fontspec}
\usepackage[sc,osf]{mathpazo}
\usepackage{libertine}
\usepackage{fontawesome5}
\usepackage{booktabs}

\setmainhangulfont{NanumMyeongjo}[
    Path={./fonts/},
    Extension={.ttf},
    UprightFont=*,
    BoldFont=NanumMyeongjoBold,
]
\setmonofont[LetterSpace=-5.0,Scale=0.75]{Libertinus Mono}

\def\myname{Chiwan Park}
\def\myemail{chiwanpark@hotmail.com}
\def\myhomepage{https://chiwanpark.com}

\hypersetup{%
  pdftitle={Curriculum Vitae},
  pdfauthor={\myname},
  pdfsubject={},
  pdfkeywords={\myname, Curriculum Vitae},
  plainpages=false,
  bookmarksnumbered=true,
  pdfstartview={FitV},
  colorlinks={true},
  urlcolor={NavyBlue},
  pdfborder={0 0 0},
  pdfcreator={LuaTeX}
}

\pagestyle{myheadings}
\markright{\myname}
\thispagestyle{empty}

\usepackage{sectsty}
\sectionfont{\rmfamily\bfseries\Large\sectionrule{0ex}{0pt}{-1ex}{0.5pt}}

\setlength{\parskip}{-0.2em}
\setlength\parindent{0em}
\titlespacing{\section}{0pt}{\parskip}{-\parskip}

\newenvironment{entry}{
  \begin{list}{}{
    \setlength{\leftmargin}{0em}
    \setlength{\itemsep}{0.25em}
    \setlength{\parskip}{0pt}
    \setlength{\parsep}{0.25em}
  }
}{
  \end{list}
}
\setlist[itemize]{leftmargin=1em}
\let\orighref\href
\renewcommand{\href}[2]{\orighref{#1}{#2\ {\scriptsize\faIcon{external-link-alt}}}}

\begin{document}

% HEADER
\begin{minipage}[t]{0.375\textwidth}
    \vspace{0pt}
    \Huge
    \textsc{\myname}
\end{minipage}
\hfill
\begin{minipage}[t]{0.6\textwidth}
  \vspace{0pt}
  Applied Machine Learning Engineer at Kakao\\
  {\scriptsize\faIcon{envelope}}\ \orighref{mailto:\myemail}{\myemail}\ \ {\scriptsize\faIcon{globe-americas}}\ \orighref{\myhomepage}{\myhomepage}
\end{minipage}

% RESEARCH INTERESTS
\section*{Research Interests}
\begin{entry}
  \item \textbf{Conversational Agents:} building production-grade conversational agents that can understand and respond to user queries in natural language, leveraging workflow graphs and large language models.
  \item \textbf{Machine Learning on Graphs:} designing algorithms for analyzing and discovering useful information from real-world graphs by exploiting properties of the graphs.
\end{entry}

% EDUCATION
\section*{Education}
\begin{entry}
  \item \textbf{Master of Science}, Computer Science and Engineering\hfill\textsc{Mar. 2016 -- Feb. 2018}\\
        Seoul National University\\
        Thesis: \href{https://s-space.snu.ac.kr/bitstream/10371/141560/1/000000151119.pdf}{Pre-partitioned Matrix-Vector Multiplication for Scalable Graph Mining}\\
        Advisor: U Kang
  \item \textbf{Bachelor of Science}, Earth System Sciences\hfill\textsc{Mar. 2010 -- Feb. 2016}\\
        \textbf{Bachelor of Engineering}, Computer Science (Double Major)\\
        Yonsei University
\end{entry}

% WORK EXPERIENCES
\section*{Work Experiences}
\begin{entry}
  \setlength\itemsep{0.1em}
  \item \textbf{Applied Machine Learning Engineer (Full-time)} at Kakao\hfill\textsc{Apr. 2018 -- Present}
  \begin{itemize}
    \item Involved in building \href{https://mate.kakao.com/}{AI Mate}, a conversational agent designed to deliver personalized recommendations and assistance across various Kakao services.
    \begin{itemize}
      \item Designed and implemented a language model fine-tuning technique for workflow graph-based agents with multiple system prompts, crucially ensuring they function without interfering with each other.
      \item Implemented a versatile LLM inference system leveraging multiple backend engines, including vLLM, SGLang, and TensorRT-LLM.
      \item Published our approach as a paper at \href{https://arxiv.org/abs/2505.23006}{ACL 2025 (Industry Track)}
    \end{itemize}
    \item Built and maintained machine learning applications for several Kakao services, including Daum, Kakao Webtoon, KakaoTalk Gift, ShoppingHow, and Piccoma.
    \item Led a research unit of 10+ members, responsible for developing and maintaining machine learning applications across various Kakao services (\textsc{Oct}. 2019 - \textsc{Dec}. 2023).
    \begin{itemize}
      \item Mentored the members, resulting in significant key metric improvements across several services.
      \item Our research results have been presented at a variety of academic and industrial venues, including \href{https://dl.acm.org/doi/10.1145/3556702.3556851}{RecSys'22 Challenge}, \href{https://amazonkddcup.github.io/papers/0620.pdf}{KDD Cup'22}, \href{https://speakerdeck.com/chiwanpark/challenges-in-real-world-recommender-systems}{KCC 2022}, \href{https://dl.acm.org/doi/10.1145/3539618.3591934}{SIGIR'23}, and multiple if(kakao) events (\href{https://if.kakao.com/2019/program?sessionId=dce0dd84-d054-4b80-8013-b3d58f61bbe8}{2019}, \href{https://if.kakao.com/2020/session/125}{2020}, \href{https://if.kakao.com/2021/session/27}{2021}, \href{https://if.kakao.com/2022/session/8}{2022}).
    \end{itemize}
    \item Contributed to the recruitment process, conducting 50+ interviews and operating internship programs.
    \begin{itemize}
      \item Our internship programs became the foundation of the Kakao internship program.
    \end{itemize}
  \end{itemize}
\end{entry}

% PUBLICATIONS
\section*{Publications}
\begin{entry}
  \item \textbf{A Practical Approach for Building Production-Grade Conversational Agents with Workflow Graphs}\\
  \underline{Chiwan Park*}, Wonjun Jang*, Daeryong Kim*, Aelim Ahn, Kichang Yang, Woosung Hwang, Jihyeon Roh, Hyerin Park, Hyosun Wang, Min Seok Kim, and Jihoon Kang\\
  The 63rd Annual Meeting of the Association for Computational Linguistics (ACL), 2025 (Industry Track)
  \item \textbf{Simple and Efficient Recommendation Strategy for Warm/Cold Sessions for RecSys Challenge 2022}\\
  Hyunsung Lee, Sungwook Yoo, Andrew Yang, Wonjun Jang and \underline{Chiwan Park}\\
  RecSys Challenge Workshop at the 16th ACM Conference on Recommender Systems (RecSys), 2022
  \item \textbf{FlexGraph: Flexible partitioning and storage for scalable graph mining}\\
  \underline{Chiwan Park}, Ha-Myung Park and U Kang\\
  PLoS ONE 15(1): e0227032
  \item \textbf{PegasusN: A Scalable and Versatile Graph Mining System}\\
  Ha-Myung Park, \underline{Chiwan Park}, and U Kang\\
  The 32nd AAAI Conference on Artificial Intelligence (AAAI) 2018 (Demo Paper)
  \item \textbf{A Distributed Vertex Rearrangement Algorithm for Compressing and Mining Big Graphs}\\
  Namyong Park, \underline{Chiwan Park}, and U Kang\\
  Journal of KIISE (Domestic), Vol. 43, No. 10, pp. 1131-1143, 2016.
\end{entry}

% PATENTS
\section*{Patents}
\begin{entry}
  \item U Kang, \underline{Chiwan Park}, Ha-Myung Park, Minji Yoon, \textbf{"Method and Apparatus for Scalable Graph Mining using Graph Pre-partitioning"}, Korean patent number: 10-1990735 (filed on Mar. 30, 2018, and registered on Jun. 12, 2019).
\end{entry}

% TEACHING EXPERIENCES
\section*{Teaching Experiences}
\begin{entry}
  \setlength\itemsep{0.1em}
  \item \textbf{Teaching Assistant}, Basic Math for Big Data at SNU BDI Academy\hfill\textsc{Summer 2017}
  \item \textbf{Teaching Assistant}, M1522.001400 Introduction to Data Mining at SNU\hfill\textsc{Spring 2017}
  \item \textbf{Teaching Assistant}, M1522.000900 Data Structures at SNU\hfill\textsc{Fall 2016}
\end{entry}

% TALKS
\section*{Talks}
\begin{entry}
  \item \href{https://speakerdeck.com/chiwanpark/challenges-in-real-world-recommender-systems}{\textbf{Challenges in Real-world Recommender Systems}}\hfill\textsc{Jun. 2022}\\
        KCC 2022, Jeju, Korea
  \item \href{http://bit.ly/chiwanpark-ifkakao2019-new}{\textbf{상품 카탈로그 자동 생성 ML 모델 소개}}\hfill\textsc{Aug. 2019}\\
        if kakao(dev) 2019, Seoul, Korea
  \item \href{http://j.mp/d2-campus-seminar-4th-park}{\textbf{학교에선 알려주지 않는 오픈 소스 이야기}}\hfill\textsc{Feb. 2016}\\
        NAVER 4th D2 Campus Seminar, Seongnam, Korea
  \item \href{http://j.mp/ossdevconf-2015-park}{\textbf{Introduction to Apache Flink}}\hfill\textsc{Dec. 2015}\\
        Open-source Software Developer Center Conference, Seoul, Korea
  \item \href{http://j.mp/cjkossforum-2015-park}{\textbf{Lessons Learned from Open-source Activities}}\hfill\textsc{Nov. 2015}\\
        The 14th Northeast Asia OSS Promotion Forum, Tokyo, Japan
\end{entry}

\bigskip
{\small Last updated: \today}

\end{document}
