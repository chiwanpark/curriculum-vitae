\documentclass[11pt,a4paper]{article}
\usepackage[margin=0.9in]{geometry}
\usepackage{kotex}
\usepackage{hyperref}
\usepackage{enumitem}
\usepackage{lmodern}
\usepackage[utf8]{inputenc}
\usepackage[T1]{fontenc}
\usepackage[sc,osf]{mathpazo}

\def\myname{Chiwan Park}
\def\myphone{+82-10-8518-3832}
\def\myaddress{Data Mining Laboratory\\
\#519, Building 301, Seoul National University\\
1 Gwanak-ro, Gwanak-gu, Seoul,\\
Republic of Korea 08826}
\def\myemail{chiwanpark@icloud.com}
\def\myhomepage{http://chiwanpark.com}

\hypersetup{%
  pdftitle={Curriculum Vitae},
  pdfauthor={\myname},
  pdfsubject={},
  pdfkeywords={\myname, Curriculum Vitae},
  plainpages=false,
  bookmarksnumbered=true,
  pdfstartview={FitV},
  colorlinks={false},
  pdfborder={0 0 0},
  pdfcreator={LuaTeX}
}

\pagestyle{myheadings}
\markright{\myname}
\thispagestyle{empty}

\usepackage{sectsty}
\sectionfont{\rmfamily\mdseries\Large}
\subsectionfont{\rmfamily\mdseries\itshape\large}

\setlength\parindent{0em}

\renewenvironment{itemize}{
  \begin{list}{}{
    \setlength{\leftmargin}{1.5em}
    \setlength{\itemsep}{0.5em}
    \setlength{\parskip}{0pt}
    \setlength{\parsep}{0.25em}
  }
}{
  \end{list}
}
\setlist[enumerate]{itemsep=0.25em}

\begin{document}

% HEADER
\par{\Huge \textsc{\myname}}

\bigskip

\begin{minipage}[t]{0.495\textwidth}
  \myaddress
\end{minipage}
\begin{minipage}[t]{0.495\textwidth}
  Phone: \myphone \\
  Email: \href{mailto:\myemail}{\tt \myemail} \\
  Homepage: \href{\myhomepage}{\tt \myhomepage}
\end{minipage}

% RESEARCH INTERESTS
\section*{Research Interests}
\begin{itemize}
  \item \textbf{Large-scale Graph Processing:} designing algorithms for analyzing and discovering useful information from real-world graphs on distributed systems.
  \item \textbf{Scalable Machine Learning:} developing machine learning algorithms in order to handle large-scale dataset on distributed systems.
\end{itemize}

% EDUCATION
\section*{Education}
\begin{itemize}
  \item \textbf{Master of Science}, Computer Science and Engineering\hfill\textsc{Mar. 2016 -- Feb. 2018} (Expected)\\
        Seoul National University\\
        Advisor: U Kang
  \item \textbf{Bachelor of Science}, Earth System Sciences\hfill\textsc{Mar. 2010 -- Feb. 2016}\\
        \textbf{Bachelor of Science in Engineering}, Computer Science (Double Major)\\
        Yonsei University
\end{itemize}

% HONORS AND AWARDS
\section*{Honors and Awards}
\subsection*{Scholarship}
\begin{itemize}
  \item \textbf{Horor Scholarship}\hfill\textsc{Feb. 2015}\\
        Yonsei University
  \item \textbf{Sinchon Market Supporters Scholarship}\hfill\textsc{Nov. 2014}\\
        Yonsei University
  \item \textbf{Scholarship for Basic Human Resources of Geoscience}\hfill\textsc{Sep. 2012}\\
        Ministry of Knowledge and Education of Korea
\end{itemize}
\subsection*{Fellowship}
\begin{itemize}
  \item \textbf{Global Open Frontier Program}\hfill\textsc{Dec. 2013 -- Dec. 2015}\\
        National IT Industry Promotion Agency of Korea
  \item \textbf{Support for Open Source Software Development}\hfill\textsc{Jul. 2013 -- Nov. 2013}\\
        National IT Industry Promotion Agency of Korea
\end{itemize}
\subsection*{Awards}
\begin{itemize}
  \item \textbf{3rd Prize} of Korea Big Data Analysis Contest 2013\hfill\textsc{Nov. 2013}\\
        National Information Society Agency of Korea
  \item \textbf{3rd Prize} of SK Telecom Mobile Web Application Contest 2013\hfill\textsc{Dec. 2013}\\
        SK Telecom
\end{itemize}

% WORK EXPERIENCE
\section*{Work Experience}
\begin{itemize}
  \item \textbf{Software Engineer (Part-time)} in Anbado Video\hfill\textsc{Jan. 2014 -- Apr. 2015}\\
        Anbado Video is a startup company creating new experience of video. I developed a back-end of SNS service with videos called modac.TV. I used Python, Flask, SQLAlchemy, PostgreSQL and Google Cloud Platform. I also performed data analysis using Apache Pig.
  \item \textbf{Software Engineer (Part-time)} in Cloudine\hfill\textsc{Jul. 2013 -- Dec. 2013}\\
        Cloudine is a company developing data processing platform and consulting set-up of infrastructure for large-scale data processing. I tested performance of Apache Mahout and developed near real-time recommendation service based on Apache Mahout, mongoDB and Spring Framework.
\end{itemize}

% EXTRA ACTIVITY
\section*{Extra Activity}
\begin{itemize}
  \item \textbf{Committer} of Apache Flink\hfill\textsc{Nov. 2014 -- Present}\\
        Apache Flink is an open-source platform for scalable batch and stream data processing. I have contributed to Apache Flink since Nov. 2014. I was invited as a committer of Apache Flink in Jun. 2015. I have learned a lot of basic concepts about distributed processing and scalable machine learning algorithm from this project. I use Java and Scala to contribute to this project.
  \item \textbf{Contributor} of Flamingo Project\hfill\textsc{Jul. 2013 -- Dec. 2014}\\
        Flamingo Project is an open-source software for providing full control of Hadoop Ecosystem in web-based user interface. I have participated since Jul. 2013. I implemented Apache Hive and Apache Sqoop interface of Flamingo 1. I use Java and Javascript to contribute to this project.
  \item \textbf{3rd Trainee} of Software Maestro\hfill\textsc{Jul. 2012 -- Jun. 2013}\\
        Software Maestro is a greatest mentoring program to train excellent software programmers in Korea. This program is backed by Ministry of Science, ICT and Future Planning. I learned many key ideas of MapReduce programming model and Hadoop ecosystem such as HDFS, Hadoop MapReduce, Pig, and Hive. I developed a search engine prototype using TF-IDF weighting factor based on MapReduce programming model.
\end{itemize}

% CONFERENCE PRESENTATIONS
\section*{Presentations}
\begin{itemize}
  \item \href{http://j.mp/d2-campus-seminar-4th-park}{\textit{학교에선 알려주지 않는 오픈 소스 이야기}}\hfill\textsc{Feb. 2016}\\
        4th D2 Campus Seminar, Pangyo, Korea
  \item \href{http://j.mp/cjkossforum-2015-park}{\textit{Lessons Learned from Open-source Activities}}\hfill\textsc{Nov. 2015}\\
        The 14th Northeast Asia OSS Promotion Forum, Tokyo, Japan
  \item \href{http://j.mp/ots-2014-park}{\textit{쓰기 쉬운 Hadoop 기반 빅데이터 플랫폼 아키텍처 및 활용 방안}}\hfill\textsc{Mar. 2014}\\
        Open Technet Summit 2014, Seoul, Korea
\end{itemize}

% TECHNICAL SKILLS
\section*{Technical Skills}
\subsection*{Programming Languages}
\begin{itemize}
  \setlength{\itemsep}{0.25em}
  \item Java, Python (Advanced)
  \item Scala, C, C++, Javascript, \LaTeX\ (Intermediate)
  \item MATLAB, Fortran (Experienced)
\end{itemize}
\subsection*{Databases and Data Processing Platforms}
\begin{itemize}
  \setlength{\itemsep}{0.25em}
  \item MySQL, PostgreSQL, Apache Hadoop, Apache Flink, Apache Spark (Intermediate)
  \item mongoDB, Redis (Experienced)
\end{itemize}

% REFERENCES
\section*{References}
\begin{itemize}
  \item \textbf{Prof. U Kang}\\
        Department of Computer Science and Engineering\\
        Seoul National University\\
        Seoul, Korea\\
        \href{mailto:ukang@snu.ac.kr}{\tt ukang@snu.ac.kr}
\end{itemize}

\bigskip
{\small Last updated: \today}

\end{document}